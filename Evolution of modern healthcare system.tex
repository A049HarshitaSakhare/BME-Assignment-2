\documentclass[12pt,a4paper]{article}

\usepackage{tikz}
\usetikzlibrary{calc}

\usepackage{graphicx}
\graphicspath{{Images/}}


\usepackage[T1]{fontenc}
\usepackage{tgbonum}



\begin{document}
\begin{tikzpicture}
[remember picture, overlay]  \draw[line width=3pt]  ($(current page.north west)+(0.5in,-0.5in)$)  rectangle ($(current page.south east)+(-0.5in,0.5in)$);


\end{tikzpicture}


\hspace{10cm}

\begin{center}
{\large NATIONAL INSTITUTE OF TECHNOLOGY RAIPUR}
\end{center}

\hspace{10cm}


\begin{figure}
\centering
\includegraphics[scale=0.15]{nitrr.jpg}
\end{figure}



\hspace{11cm}

\begin{center}
{\fontfamily{qcr}\selectfont \large Submitted By:- Harshita Sakhare }
\end{center}




\hspace{10cm}

\begin{center}
\textit{\large Roll.No:- 21111049}
\end{center}



\hspace{10cm}


\begin{center}
\textsf{\large Submitted To:- Prof. Saurabh Gupta}
\end{center}


\hspace{10cm}

\begin{center}
\textbf{Assignment-2}
\end{center}


\hspace{10cm}


\begin{center}
\textbf{\large Evolution of Modern Health Care System}
\end{center}

\begin{center}
\textit{Date: 3 February 2022}
\end{center}

\clearpage
\tableofcontents
\clearpage


\section{Introduction to Modern Healthcare}
{\large Modern Healthcare has evolved and become more\\ focused on prevention. Preventative efforts are in place to reduce an eradicate disease, support overall physical and mental health, and educate patients and families to promote safety. Due to this emphasis on prevention as well as our obsession with technology and convenience, it only makes sense that trends and new technology in healthcare focus on providing patients with more access to healthcare so that they can prevent issues before they turn into huge headaches. Modern trends in healthcare includes:}

\begin{itemize}
{\large \item Digital check-ups
\item Smartphone applications to check your symptoms
\item Virtual visits with your physician}
\end{itemize}


\begin{figure}[h]
\centering
\includegraphics[scale=1]{modern healthcare.jpg}
\end{figure}



\clearpage




\section{Electroencephalogram (EEG)}
{\large An Electroencephalography(EEG) sensor is an electronic device that can measure electrical signals of the brain. EEG sensors typically measure the varying electrical\\ signals created by the activity of large groups of neurons near the surface of the brain over a period of time. They work by measuring the small fluctuations in electrical current between the skin and the sensor electrode,\\ amplifying the electrical current, and performing any\\ filtering, such as bandpass filtering.\\ Innovations in the field of medicine began in the early 1900s, prior to which there was little innovation due to the uncollaborative nature of the field of medicine. Innovation in diagnosis and treatment came from\\interdisciplinary advances in the applied sciences, such as those of physics and chemistry. Once such\\ innovation was the discovery of the small electrical\\ currents produce by the brain and other organs.\\ Measurement of electrical activity, such as in EEG, was not performed until after 1903, when the technique to measure the electrical activity of the heart was\\ discovered by Willem Einthoven. This measurement\\ technique was extended to the brain to extract the EEG signal.}



\subsection{History of Electroencephalogram (EEG)}




\begin{figure}[h]
\centering
\includegraphics[scale=0.3]{eeg.jpg}
\caption{Hans Berger EEG Model}
\end{figure}




{\large Hans Berger (1873-1941), a German psychiatrist, recorded the first human (EEG), charting electrical activity of his son's brain. He also invented the first electroencephalogram device in 1924. In 1934, Fisher and Lowenback first demonstrated epileptiform spikes. In 1935, Gibbs, Davis, and Lennox described interictal epileptiform discharges and 3-Hz spike-wave patterns during clinical seizures.}



\subsection{EEG Device Design Technology}

\subsubsection{Wired and wireless communications}

{\large Wired and wireless EEG headsets transfer a data to a computer via a cable, wireless or bluetooth connection, respectively. Wired EEG connections are more stable and often can transfer more data in a given time, but do not offer the freedom of movement provided by wireless connections. One of the main drawbacks of wireless EEG headsets is that, during the capture of brain data, the headset may loose its wireless connectivity and not record the data. Regardless of the connection type, the movement of cables and electrodes can cause artifacts in the EEG signal, as it can disrupt the connection between the electrodes and the scalp.}


\subsubsection{Electrode connection}


{\large EEG devices require a consistent electrical connection between the individual electrodes and the scalp of the individual wearing the device. This can be achieved in a variety of ways, some of which are listed below.}


\subsubsection{Wet EEG devices}

{\large There are different types of wet EEG devices discussed below:}


\begin{itemize}

\item Soft gel-based:Using this connection, electrodes connect with the scalp by applying conductive gel into the pocket of each electrode. After completion of an experiment, it is necessary to clean the headset by removing the gel and cleaning the electrode. This is often done with alcohol because of its evaporative property.
\begin{figure}[h]
\centering
\includegraphics[scale=0.5]{eeg.2.jpg}
\caption{Gel based EEG Headset}
\end{figure}

\item Saline solution:Some of the EEG  headsets require a conductive gel help make low-impedance electrical contact between the skin and the sensor electrode. EEG headsets that have this technology connect electrodes by applying saline to each electrode.

\begin{figure}[h]
\centering
\includegraphics[scale=0.3]{saline solution.jpg}
\caption{Saline solution EEG Headset}
\end{figure}


\item Dry:Dry EEG devices do not use any gel or saline to connect the electrodes with the scalp, which makes it easier to record EEG data without the help of a trained technician.\\Furthermore, its setup time is considerably shorter than wet headsets.


\begin{figure}[h]
\centering
\includegraphics[scale=0.05]{dry eeg.jpg}
\caption{Dry EEG}
\end{figure}

\item Others: Some EEG sensor connections types do not fit cleanly into either of these two categories. Cnductive solid gel materials, such as those produced by Enobio, have also been used successfully in EEG devices.

\end{itemize}

\subsubsection{Difference between Dry and Wet devices}

Conventional EEG has been recorded with "wet" electrodes that use a layer of conductive gel or paste to increase conductivity between the electrodes and the test subject's skin. Dry electrodes,however,do not require conductive gel and are set up much faster.\\In January of 2019, researchers at the University of California, The Otto von Guericke University of Magdeburg,and The Hebrew University of Jerusalem performed a comparative analysis of the signal quality of dried wireless and wet wire EEG devices,and concluded that the quality of wireless dry devices is significantly comparable with the wired wet. Although some researchers observed that, for those activities that demand body movement like running/walking,wired wet sensors showed better performance. This seems to indicate that wet sensors may be more resistant to movement artifacts, although more research needs to be conducted to fully understand which technology can provide more reliable data.



\subsubsection{Electrode placement standards}

The T3,C3,Cz,C4,and T4 electrodes are placed at marks made at intervals of 10 percent,20 percent,20 percent, 20 percent, 20 percent and 10 percent, respectively, measured  across the top of the head. Skull circumference is measured just above the ears (T3 and T4), just above the bridge of the nose(at Fpz), and just above the occipital point(at Oz).

\begin{figure}[h]
\centering
\includegraphics[scale=0.2]{eeg 10-20 system.jpg}
\caption{10-20 system (EEG)}
\end{figure}

\subsubsection{EEG device with several sensors}

Some EEG devices are designed to capture both psychological(brain)and physiological(blood pressure, muscle activity, heart rate, etc.)data. These devices have one or more extra channels for capturing physiological signals such as electrocardiogram(ECG), Electrooculography(EOG), Photoplethysmogram(PPG), and Electromyography(EMG).\\

ECG  sensors record the heart's response during resting or\\ physical activity.
 
 
 
 EOG sensors measur human eye movements.
 
 
 
 
  PPG sensors monitor blood volume changes.
  
  
   EMG sensors collect muscle activity data.
   
   
\hspace{1cm}
   
   
   Some EEG devices are equipped with motion sensors such as gyroscopes and accelerometers to capture head and body motion data. These can be used to measure,e.g., orientation, acceleration,and speed.


\section{Application of EEG}


\subsection{Brain Computer Interfaces (BCI)}

A relatively new but emergent field for EEG is brain-computer interfaces. Today, we know in much more detail which brain areas are active when we perceive stimuli, when we prepare and execute\\ bodily movements, or when we learn and memorize things.This gives rise to very powerful and targeted EEG applications to steer devices using brain activity. This can, for instance, help paralyzed patients steer there wheel chair or move a cursor on a screen,but BCI technology is also used for military scenarios where soldiers are equipped with and exoskeleton and EEG cap, allowing them to move, lift and carry very heavy items simply based on brain activity.

The most common BCI applications are listed below:

\subsubsection{Autonomous navigation of digital or mechatronic devices:}

\begin{itemize}
\item Real-time teleoperation robotic body parts.
\item Controlling and directing a robot, drone, dashboard of a vehicle, or a miniature or semi-automated car.
\item Monitoring and controlling sensors inside of smart houses.
\end{itemize}

\subsubsection{Helping people with disabilities or motor activity impairment:}

\begin{itemize}
\item Control of mobile phone apps using eyewinks.
\item Directing electrical wheelchair movement.
\item Control of artificial body part such as prosthetic hand or arm.
\item Recognizing a patient's attempt to move their body, e.g.,stroke and brain injury.
\item Post-stroke motor rehabilitation using VR.
\item Controlling a robot using body gestures.
\item Mind-controlled dialing systems.
\item Speech recognition system for people with speech disability.
\item Mouse cursor control using imagined hand movement.
\item Gaze controller for patients with neurodegenerative diseases.
\end{itemize}

\subsubsection{Neurogaming and entertainment:}

\begin{itemize}
\item Controlling a video game or virtual reality (VR) environment using body gesture and eye movement.
\item Controlling fiber optic clothes or dresses.
\end{itemize}


\subsection{Neurology:}
Real-time EEG signals can be use to provide immediate information about brain-wave activities. EEG data have been applied for diagnosing and predicting many abnormal brain diseases and cognitive impairments, listed below:
\begin{itemize}
\item Epilepsy
\item Parkimsom's disease
\item Memory problems like Alzheimer's.
\item Language impairments such as Dyslexia.
\item Attention Deficit Hyperactivity Disorder(ADHD).
\item Seizures
\item Schizophrenia
\item Autism in adults and children.
\item Sleep disorders and insomnia.
\item Anxiety
\item Post-traunatic stress disorder
\item Huntington's disease
\item Multiple sclerosis diagnosis
\item Anyoprophic lateral sclerosis
\item Traunatic brain injury(TBI)
\item Coma
\item Level of consciousness
\item Neurosurgery
\end{itemize}

\subsection{Neuroscience Research:}

Neuroscience attempts to understand the functionality of the nervous system. It allows clinical or non-clinical researches to get an idea about how the brain acts when human experience different emotional states and how the brain works in various mental states. Researchers have applied EEG devices in their studies in the below fields:

\subsubsection{Cognitive Neuroscience:}

\begin{itemize}
\item Measuring cognitive load
\item Detecting differences between brain wave activity during suicidal and non-suicidal states. Understanding brain activity during insight (insight is a moment where human understands how to solve a puzzle or gains knowledge).
\item Analyzing brain workload during decision making or learning a new task.
\item Studying sleep pattern.
\end{itemize}


\subsubsection{Behavioural Neuroscience:}

\begin{itemize}
\item Changing the work place light and measuring brain alertness status.
\item Measuring drowsiness or sleep detection for drivers and pilots.
\item Measuring mental workload of deaf children exposed to a noisy environment during a word recognition task.
\item Determining surgeon stress level while performing surgery.
\item Identifying and reducing stress level.
\item Environmental psychology
\end{itemize}


\subsubsection{Neurophysiology:}

\begin{itemize}
\item Measuring changes in brain after drinking alcohol.
\item Detecting fatigue
\end{itemize}

Neuroscience can also be applied to understanding human emotion in VR with/without the ability to touch the environment by displaying various type of media, such as:

\begin{itemize}
\item Real world or VR pictures
\item Images of nature and city environments
\item TV advertisements
\item Auditory stimuli
\item Multimedia, along with memory recall and dreams
\end{itemize}

\subsection{Neuromarketing:}

In the field of neuromarketing, economists use EEG research to detect brain processes that drive consumer decisions, brain areas that are active we purchase a product/service, and mental states that the respective person is in when exploring physical or virtual stores. Nowadays, studies can be conducted in mobile setups to gain insights into shopping habit and decision-making in real-world scenariors. 

\subsection{Biometrics:}

Recognizing and distinguishing people using physiological or behavioral features such as fingerprint, voice, face, iris, gaze, gait and/or posture is called biometrics. Study show that EEG data can provide information and individuals differences. Recently, cognitive and emotional brain status has been utilized for biometrics, meaning EEG data are used to identify people. The main ideas behind why EEG-based biometric systems have received more attention recently relate to privacy compliance and robustness to spoofing attacks, as well as universality.


\section{EEG in 21st Century} 


As one of the first common use of brain computer interface technology, EEG neurofeedback has been in use for several decades. The year 1998 marked a significant development in the field of brain mapping when researcher Philip Kennedy implanted the first brain computer interface object into a human being. Brain-computer\\ interface BCI's technology is used in EEG since that time,\\ sometimes called brain-machine interfaces(BMI's), are one of the most common application of EEG. BCI's use real-time EEG data to control and direct mechanical and electronic devices. EEG headsets detect brain activity through electrodes placed in an array along the user or researcher subject's scalp. Wet electrodes use a conductive gel, saline fluid or other material to improve signal quality. This\\ ensures the EEG device captures high-quality data. Today wearable EEG  is envisioned as the evolution of ambulatory EEG units from the bulky, limited lifetime devices available today to small devices present only on the head that can record EEG  for days, weeks, or months at a time.

When EEG was invented it was a bulky device which we can't carry every where but today the wearable and wireless EEG\\ headsets are available. The advancement of wireless EEG sensor technology is rapidly changing the way we interact with the world. Through the use of EEG headsets, EEG sensors measure the brain's electrical activity, or brainwaves. In early research, EEG testing was invasive and complex. It typically involved the use of silver needles and  electrode attachments to the scalp, and had to be done in hospitals or research settings.


Today, wireless EEG headsets allow readings to be done completely non-invasively from the comfort of your home. They are simple and comfortable to wear, and are powered by AAA battery. They're also a very affordable. Modern EEG sensors translate\\ brainwaves into action, which can be harnessed and used with apps that focus on health and wellness, education, and entertainment.

\begin{figure}[h]
\centering
\includegraphics[scale=0.15]{hs.jpg}
\caption{EEG Wireless headset with app}
\end{figure}

\begin{figure}[h]
\centering
\includegraphics[scale=0.3]{headset.jpg}
\caption{Neurosky headset with its app}
\end{figure}

\clearpage




\section{New technologies implemented in healthcare in previous 3 years:}

\textbf {In year 2019 top technologies:}
\begin{itemize}
\item Personalized Medicine
\item Telehealth
\item Blockchain
\item AI and Machine Learning
\item Cancer immunotherapy
\item 3D Printing
\item Augmented reality and Virtual reality
\item Robotic Surgery
\item Quantum Computing
\item The internet of things
\end{itemize}



\textbf {In year 2020 top technologies:}
\begin{itemize}
\item Advanced Telemedicine 
\item New Methods of Drug Development
\item Data-Driven Healthcare
\item Nanomedicine
\item 5G-Enabled Devices
\item Tricorders
\item Healthcare's Digital Assistants
\item Smarter Pacemakers
\item A Lab on a Chip
\item Wearables with a purpose
\end{itemize}

\clearpage

\textbf{In year 2021 emerging technologies:}
\begin{itemize}
\item More strategic and agile supply chains
\item Coopetition as a viable strategy
\item Patient Consumerization 
\item Personalization of care
\item Workforce Diversity and safety
\item Virtual care
\item Artificial Intelligence and automation 
\item Revenue Diversification

\end{itemize}


















\section{Conclusion and Future Research:}

EEG devices are quickly becoming less expensive and more\\ accessible to the open market, which should allow for more\\ commercial and personal use of data. Because of their wide spread availability, many consideration should be made before a decision is made to purchase and use a device. Along with price, other factors that need to be considered are the battery life of an EEG device, available software for data analysis, and common uses of the device in research areas, especially where your own research may apply.


For future research, areas like biometrics and neuro-marketing currently have very little related research, and thus may be good avenues for further study.

\clearpage 



\section{Reference}


\begin{itemize}
\item Wikipedia
\item mdpi.com
\item imotions.com
\item neurosky.com
\end{itemize}










\end{document}